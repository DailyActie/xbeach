\chapter{ Compiling the model}

To compile the source code you can download the source code from the subversion repository, located at: https://repos.deltares.nl/repos/XBeach/trunk

The source code contains three different build environments:

GNU autotools, tested with gfortran $>$=4.2, tested with ubuntu, OSX and cygwin with windows XP.

Visual studio 2008, tested with intel fortran 11, tested with Windows XP

Visual studio 6, tested with Compaq fortran 6.6, tested with Windows XP

We expect it to compile with only minor problems under other compilers and/or under Linux, since only standard Fortran 90/95 is used. Once an executable has been created, it will be called xbeach or xbeach.exe under windows. 

Building with MPI support

XBeach can be built with support for mpi. To build with mpi under windows, make sure you have mpi installed in c:\textbackslash program files\textbackslash mpich2. Select the appropriate configuration in visual studio (mpi debug or mpi release) and use rebuild all to make a mpi enabled executable.

To build a mpi executable use ./configure --with-mpi \&\& make to build an executable with mpi support. 

XBeach is tested with openmpi on linux and OS X and with mpich2 on all platforms.

Building with NetCDF support

XBeach can be built with support for netcdf. To build with netcdf under windows, select the appropriate configuration in visual studio (netcdf) and use rebuild all to make a netcdf enabled executable. 

To build a netcdf executable under linux make sure you have netcdf 4.1 or higher installed. It should be automatically found by the configure script and compiled. 

The combination of netcdf and mpi has not yet been tested. 

For more details see the README that come with the source code. 

\begin{tabular}{|p{1.5in}|p{1.5in}|p{1.5in}|} \hline 
XBeach Model Description and Manual & Z4175 & June 21, 20 \\ \hline 
&  &  \\ \hline 
\end{tabular}

D:\textbackslash mccall\textbackslash XBeach\textbackslash XBeach\_repository\textbackslash trunk\textbackslash doc\textbackslash xbeach\_manual.doc

%%% Local Variables: 
%%% mode: latex
%%% TeX-master: "xbeach_manual"
%%% End: 
